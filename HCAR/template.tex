\documentclass{scrreprt}
\usepackage{paralist}
\usepackage{graphicx}
\usepackage[final]{hcar}

\begin{document}

\begin{hcarentry}{Idris}
\report{Edwin Brady}
\status{active development}
%\participants{(PARTICIPANTS OTHER THAN MYSELF)}% optional
\makeheader

Idris is an experimental language with full dependent types.
Dependent types allow types to be predicated on values, meaning that
some aspects of a program's behaviour can be specified precisely in
the type. The language is closely related to Epigram and Agda.
It is available from \url{http://www.idris-lang.org}, and there is a
tutorial at \url{http://www.cs.st-andrews.ac.uk/~eb/Idris/tutorial.html}.

Idris aims to provide a platform for realistic programming with
dependent types. By realistic, we mean the ability to interact with
the outside world and use primitive types and operations, to make a
dependently typed language suitable for systems programming. This
includes networking, file handling, concurrency, etc.
Idris emphasises programming over theorem proving, but nevertheless
integrates with an interactive theorem prover. It is compiled, via C,
and uses the Boehm-Demers-Weiser garbage collector.

One goal of the project is to show that Idris, and dependently typed
programming in general, can be efficient enough for the development of
real world verified software. To this end, Idris is current being used
to develop a library for verified network protocol implementation,
with example applications.

\FurtherReading
  \url{http://www.idris-lang.org/}
\end{hcarentry}

\end{document}
